%!TEX root = ./template-skripsi.tex
%-------------------------------------------------------------------------------
% 								BAB I
% 							LATAR BELAKANG
%-------------------------------------------------------------------------------

\chapter{LATAR BELAKANG}

\section{Latar Belakang }

Sumber daya manusia merupakan salah satu unsur dalam suatu organisasi atau perusahaan. Sumber daya manusia yang terdapat dalam suatu organisasi memiliki latar belakang pendidikan yang berbeda-beda. Dengan latar belakang yang berbeda tersebut maka diperlukan manajemen yang baik agar tercapai tujuan organisasi. Sumber daya manusia dalam organisasi sangat penting bagi keberhasilan mencapai tujuan. Pegawai tidak dipandang hanya sebagai modal atau biaya (\textit{expense}), tetapi pegawai dianggap sebagai salah satu bentuk \textit{organizational resource} yang dapat meningkatkan kompetitif organisasi. Oleh karena itu, agar pegawai dapat menjadi sumber daya utama dan menentukan dalam mensukseskan tugas-tugas, maka harus dikembangkan kemampuannya atau mampu memecahkan setiap permasalahan yang ada dalam organisasi tersebut.

Untuk mengatasi persoalan dalam sebuah organisasi, yang mana seorang karyawan dihadapkan oleh sejumlah tugas dan tanggung jawab yang besar serta tuntutan akan peran profesinya, dan di lain pihak adanya keterbatasan yang dimiliki oleh karyawan itu sendiri maupun keterbatasan akan apa yang diharapkan untuk diperoleh dari profesinya, sangat dibutuhkan perilaku ekstra peran dari para karyawan yang dikenal sebagai \textit{ Organizational Citizenship Behavior} (OCB).

Menurut \textcite{organ2006} OCB sebagai perilaku individual yang bersifat bebas (\textit{discretionary}), yang tidak secara langsung dan eksplisit mendapat penghargaan dari sistem imbalan formal, dan yang secara keseluruhan (agregat) meningkatkan efisiensi dan efektifitas fungsi-fungsi organisasi. Bersifat bebas dan sukarela, karena perilaku tersebut tidak diharuskan oleh persyaratan peran atau deskripsi jabatan yang secara jelas dituntut berdasarkan kontrak dengan organisasi, melainkan sebagai pilihan personal. OCB merupakan perilaku positif orang-orang yang ada dalam organisasi, yang terekspresikan dalam bentuk kesediaan secara sadar dan sukarela untuk bekerja \parencite{Erratt2013}.

Meningkatnya perilaku OCB tidak lepas dari peran serta sumber daya manusia yaitu pegawai dalam organisasi perusahaan. Menurut \textcite{miramargaretha2012} mengingat pentingnya aspek manusia bagi organisasi, maka peran seorang pemimpin pun tidak kalah pentingnya. Keputusan dan kebijakan yang dibuat oleh seorang pemimpin diharapkan tidak saja mempengaruhi keberhasilan organisasi, tetapi juga perilaku semua karyawannya. Dalam penelitian Panyaruwe (2011) menyatakan bahwa pemimpin tersebut berfungsi untuk menggerakkan para pengikut (\textit{follower}) agar mereka mau mengikuti atau menjalankan apa yang diperintahkan dan dikehendaki pemimpin.

Lorem ipsum dolor sit amet, consectetuer adipiscing elit, sed diam nonummy nibh euismod tincidunt ut laoreet dolore magna aliquam erat volutpat. Ut wisi enim ad minim veniam, quis nostrud exerci tation ullamcorper suscipit lobortis nisl ut aliquip ex ea commodo consequat \parencite{Raluca2008}.

\section{Rumusan Masalah}
Habeo perfecto in sea. Ea deleniti gloriatur pri, paulo mediocrem incorrupte sea ei. Ad mollis scripta per. Incorrupte sadipscing ne mel. Mel ex nonumy malorum epicurei. Ne per tota mollis suscipit. Ullum labitur vim ut, ea dicit eleifend dissentias sit. Duis praesent expetenda ne sed. Sit et labitur albucius elaboraret. Ceteros efficiantur mei ad. Hendrerit vulputate democritum est at, quem veniam ne has, mea te malis ignota volumus.


\section{Batasan Masalah}
Batasan masalah pada penelitian ini adalah:
\begin{enumerate}
\item Lorem ipsum dolor sit amet, consectetuer adipiscing elit.
\item sed diam nonummy nibh euismod tincidunt ut laoreet dolore magna aliquam erat volutpat.
\item Ut wisi enim ad minim veniam, quis nostrud exerci tation ullamcorper suscipit lobortis nisl ut aliquip ex ea commodo consequat.
\end{enumerate}


\section{Tujuan Penelitian}
Eros reprimique vim no. Alii legendos volutpat in sed, sit enim nemore labores no. No odio decore causae has. Vim te falli libris neglegentur, eam in tempor delectus dignissim, nam hinc dictas an.


\section{Manfaat Penelitian}
Pro omnium incorrupte ea. Elitr eirmod ei qui, ex partem causae disputationi nec. Amet dicant no vis, eum modo omnes quaeque ad, antiopam evertitur reprehendunt pro ut. Nulla inermis est ne. Choro insolens mel ne, eos labitur nusquam eu, nec deserunt reformidans ut. His etiam copiosae principes te, sit brute atqui definiebas id.

Et affert civibus has. Has ne facer accumsan argumentum, apeirian hendrerit persequeris pro ex. Suscipit vivendum sensibus mea at, vim ei hinc numquam, at dicit timeam dissentiet mel. At patrioque intellegebat sea, error argumentum dissentias sea in.


\section{Keaslian Penelitian}
No per amet modo comprehensam, duo dolor dignissim ex, ancillae corrumpit intellegam vix te. Mel utinam signiferumque no, ex nec accusam accumsan. Et per inermis posidonium, qui et ornatus epicuri pertinax. In homero commodo usu, vel te habemus fuisset, id nec periculis sententiae efficiendi. Oblique sanctus intellegat at cum.


\section{Sistematika Penulisan}
\noindent
\textbf{BAB I : PENDAHULUAN}

Pada bab ini dijelaskan latar belakang, rumusan masalah, batasan, tujuan, manfaat, keaslian penelitian, dan sistematika penulisan.\\

\noindent
\textbf{BAB II : TINJAUAN PUSTAKA DAN LANDASAN TEORI}

Pada bab ini dijelaskan teori-teori dan penelitian terdahulu yang digunakan sebagai acuan dan dasar dalam penelitian.\\

\noindent
\textbf{BAB III : METODOLOGI PENELITIAN}

Pada bab ini dijelaskan metode yang digunakan dalam penelitian meliputi langkah kerja, pertanyaan penilitian, alat dan bahan, serta tahapan dan alur penelitian.\\

\noindent
\textbf{BAB IV : HASIL DAN PEMBAHASAN}

Pada bab ini dijelaskan hasil penelitian dan pembahasannya.\\

\noindent
\textbf{BAB V : KESIMPULAN DAN SARAN}

Pada bab ini ditulis kesimpulan akhir dari penelitian dan saran untuk pengembangan penelitian selanjutnya.\\

% Baris ini digunakan untuk membantu dalam melakukan sitasi
% Karena diapit dengan comment, maka baris ini akan diabaikan
% oleh compiler LaTeX.
\begin{comment}
\bibliography{daftar-pustaka}
\end{comment}
